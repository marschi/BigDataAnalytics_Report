    \section{Meetings}
        \subsection{Internal meetings}
        The team organized internal weekly meetings to discuss features, plan and assign the tasks for the following production cycle. The meeting was limited at a 2 hour mark in order to keep it productive and effective and remove time wasteful discussions. The time frame in which the meeting was set, was agreed by all team members based on each individual schedule, keeping in mind that is should offer enough working horizon before the next meeting with the supervisors.
        Internal meetings were structured in the following way:
        \begin{enumerate}
            \item Discussing feedback from the last meeting with the supervisors
            \item Adjusting the Kanban board with new tasks, or modifying the existing tasks with new requirements or details
            \item Defining the goal of the following sprint/week
            \item Picking up new tasks, discussing the details of each tasks, requirements and impediments
            \item Assigning tasks to each team member, depending on the complexity and each member’s experience with a certain technology, sometimes by forming smaller sub-teams
            \item Aggregating a list of questions for the next meeting with supervisors
            \item Summarizing the meeting in a 2-3 minutes recap version.
        \end{enumerate}
        
        \subsection{Meetings with supervisors}
        To keep the process reactive and to assure that the product is in a continuous development, meetings were organized with supervisors from the TUB, BMW and AWS. The meetings followed a cyclic pattern, being organized once a week and lasting from 30 to 60 minutes (based on each party’s availability). The structure of the meeting was also repetitive, building in the following way:
        \begin{enumerate}
            \item \textit{Progress update}. The development team starts each meeting with presenting the current progress of the project, usually by demo-ing some part of the code, other times by explaining the research and investigations done in the previous week.
            \item \textit{Q and A session}. After demo, the development teams present the impediments they identified during the working process, pointing out to some issues or asking for help or assistance is solving them. Also, any uncertainties are discussed and clarified.
            \item \textit{Feedback round}. The industrial partners evaluate the progress by commenting on the current state of the project, pointing to things that can be improved. They discuss potential new features, or directions the project can go.
            \item \textit{Planning round}. Based on the received feedback, the development team comes with a proposition on things that can be done in the following cycle.
        \end{enumerate}